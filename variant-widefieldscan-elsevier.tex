
\documentclass[preprint,12pt]{elsarticle}

%% Use the option review to obtain double line spacing
%% \documentclass[authoryear,preprint,review,12pt]{elsarticle}

%% Use the options 1p,twocolumn; 3p; 3p,twocolumn; 5p; or 5p,twocolumn
%% for a journal layout:
%% \documentclass[final,1p,times]{elsarticle}
%% \documentclass[final,1p,times,twocolumn]{elsarticle}
%% \documentclass[final,3p,times]{elsarticle}
%% \documentclass[final,3p,times,twocolumn]{elsarticle}
%% \documentclass[final,5p,times]{elsarticle}
%% \documentclass[final,5p,times,twocolumn]{elsarticle}

%% if you use PostScript figures in your article
%% use the graphics package for simple commands
%% \usepackage{graphics}
%% or use the graphicx package for more complicated commands

%% or use the epsfig package if you prefer to use the old commands
%% \usepackage{epsfig}


\usepackage[ansinew]{inputenc} 						% so I can write my Name without \"{u}
\usepackage{graphicx}
\usepackage{svn-multi} 								% to add SVN-Versioning-Info
\usepackage{caption,subfig}                         % [font = small]
\usepackage[alsoload=binary]{siunitx}		 		% typographically correct units (\milli\meter, \kilo\electronvolt), with [alsoload=binary] \SI{100}{\giga\byte}
\usepackage{booktabs}		 						% nice tables
\usepackage{tikz} 									% extremely nice drawings
\usepackage{pgfplots} 								% ditto for plots
\usepackage{todonotes}
\usepackage{amssymb}
\usepackage{lipsum}

\newcommand{\imsize}{\linewidth}
\newlength\imagewidth % needed for scalebars
\newlength\imagescale % needed for scalebars

\journal{Some tomography journal}

\begin{document}

\begin{frontmatter}

%% Title, authors and addresses

%% use the tnoteref command within \title for footnotes;
%% use the tnotetext command for theassociated footnote;
%% use the fnref command within \author or \address for footnotes;
%% use the fntext command for theassociated footnote;
%% use the corref command within \author for corresponding author footnotes;
%% use the cortext command for theassociated footnote;
%% use the ead command for the email address,
%% and the form \ead[url] for the home page:
%% \title{Title\tnoteref{label1}}

\title{Wide Field Scan Paper}

\author[ana]{David Haberth�r}
	\ead{haberthuer@ana.unibe.ch}
\author[psi]{Christoph Hinterm�ller}
	\ead{christoph.hintermueller@psi.ch}
\author[ana]{Johannes Schittny\corref{cor1}}
	\ead{johannes.schittny@ana.unibe.ch}
	\cortext[cor1]{corresponding author}
\author[psi,eth]{Marco Stampanoni}
	\ead{marco.stampanoni@psi.ch}
\address[ana]{Institute of Anatomy, University of Bern, Switzerland}
\address[psi]{Swiss Light Source, Paul Scherrer Institut, Villigen, Switzerland}
\address[eth]{Institute of Biomedical Engineering, University and ETH Z�rich, Switzerland}

\begin{abstract}
State-of-the-art synchrotron-based tomographic microscopy end-stations acquire volumetric data at micrometer level within a few minutes. It is therefore possible to investigate large objects at a high resolution by stacking several tomograms together and perform a local tomographic reconstruction.

The stacking of tomograms increases the field of view in vertical direction. The present work aims to implement the necessary acquisition protocols at the TOMCAT beamline of the Swiss Light Source to increase the field of view of the tomographic dataset in horizontal direction. It is the base for acquiring three dimensional information of samples with bigger volumens than currently supported. Different image acquisition protocols have been implemented for the end-user at the beamline.

We show that the lateral field of view of TOMCAT can be increased up to five times and provide the end-user of the beamline the possibility to acquire quality guided tomographic wide field scans of arbitrary samples in an unattended, automatic way.
\end{abstract}

\begin{keyword}
%% keywords here, in the form: keyword \sep keyword
tomography \sep SRXTM \sep keyword \sep keyword
%% PACS codes here, in the form: \PACS code \sep code

%% MSC codes here, in the form: \MSC code \sep code
%% or \MSC[2008] code \sep code (2000 is the default)
\end{keyword}

\end{frontmatter}


%% main text
%!TEX root = widefieldscan.tex
\svnidlong
{$HeadURL$}
{$LastChangedDate$}
{$LastChangedRevision$}
{$LastChangedBy$}

\ifhtml
\else
\begin{center}
	\fbox{
		\begin{minipage}{.618\columnwidth}
		The section below is versioned at \url{\svnkw{HeadURL}} (last commit @ \svnfileday.\svnfilemonth.\svnfileyear \space \svnfilehour:\svnfileminute, Revision: \svnkw{LastChangedRevision}).
		\end{minipage}
	} 
\end{center}
\fi

\section{Introduction}%
Synchrotron radiation based x-ray tomographic microscopy (SRXTM) is a powerful method for the non-destructive three-dimensional imaging of a broad kind of materials with a resolution on the sub-micrometer scale.

At TOMCAT---the beamline for TOmographic Microscopy and Coherent rAdiology experimenTs~\cite{Stampanoni2007} at the Swiss Light Source (SLS), Paul Scherrer Institute, Villigen, Switzerland---many user groups are presently working in diverse research areas, ranging from biology~\cite{McDonald2009,PerezHuerta2009}, biomedical research~\cite{Schittny2008,Tsuda2008,Heinzer2008} and paleontology~\cite{Gostling2008,Friis2007,Hagadorn2006,Donoghue2006} to material science~\cite{Gallucci2007}, geology~\cite{Carminati2007} to process engineering~\cite{Davenport2007,Vaucher2007}.

We are studying the structural alterations of the lung parenchyma throughout development using SRXTM. In particular we are interested in the development of the acini over the course of the postnatal lung development in mammals. An acinus represents the functional respiratory lung unit containing groups of alveoli, where the gas-exchange in the lung takes place and is defined as the complex of alveolated airways distal to the terminal bronchiole~\cite{Rodriguez1987}.

Until now, the investigation of the three dimensional structure of an acinus was either limited by the resolution of the imaging method (in the case of micro-computed tomography (\micro CT)) or the sample volume (in the case of SRXTM).  The available field of view of microscopy based imaging methods like synchrotron based tomographic beam lines and lab-based \micro CT stations is limited by the camera and microscope optics. Commonly a larger field of view resulting in a larger sample volume is traded against a lower magnification. Therefore, at TOMCAT the sample volume is limited to a cylinder with a diameter of \SI{1.52}{\milli\meter} and a height of \SI{1.52}{\milli\meter}, if only one tomogram is taken at a resolution of \SI{1.48}{\micro\meter}.

Samples of a diameter larger than the FOV in the direction perpendicular to the rotational axis require local tomography. During local tomography only the central cylinder where every part of the sample is included on the projectional image of every rotational angle may be reconstructed without artifacts. These artifacts are introduced due to the fact that lateral regions of the sample not included in all projectional images of every rotational angle (i.e.\ partial volume effects). Generally, a local tomography approach is not suitable for studies investigating the development~\cite{Schittny2008,Mund2008} and structure~\cite{Tsuda2008} of lungs using SRXTM, which require high resolution datasets providing a large lateral field of view.

To overcome these limitation, we developed a synchrotron radiation x-ray tomographic microscopy method which combines several tomographic scans into one large three dimensional dataset increasing the scanned volume up to 25 times.

\subsection{Enhancing the Field of View}%
\label{subsec:enhancing the field of view}%
An increase of the field of view parallel to the rotation axis of the sample can be achieved through the stacking of multiple scans on top of each other. For this protocol, the scan time is linearly increasing with the number of scans required to cover the size of the sample. Due to the restrictions implied by the sampling theorem this approach can not directly be used to increase the field of view horizontal to the rotation axis. The sampling theorem states, that we need to acquire an amount of projections $P=D\frac{\pi}{2}$ for a detector width $D$.

As a consequence, when enlarging the field of view perpendicular to the rotation axis it is necessary to record more projections at the lateral parts of the sample compared to the central parts of the sample to equally fulfill the sampling theorem. This leads to an increase in acquisition and post-processing time as compared to a standard scan. We defined and validated multiple scanning protocols to decrease the acquisition time and performed three-dimensional reconstructions of datasets with increased lateral field of view of up to \SI{7}{\milli\meter} at a voxel size of \SI{1.48}{\micro\meter}.

%% The Appendices part is started with the command \appendix;
%% appendix sections are then done as normal sections
%% \appendix

%% \section{}
%% \label{}

\begin{thebibliography}{00}

%% \bibitem{label}
%% Text of bibliographic item

\bibitem{}

\end{thebibliography}
\end{document}
