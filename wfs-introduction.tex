% !TEX root = widefieldscan.tex
\svnidlong
{$HeadURL$}
{$LastChangedDate$}
{$LastChangedRevision$}
{$LastChangedBy$}
%
\section{Introduction}
Structural alterations of the lung parenchyma throughout lung development and specifically, the development of the functional respiratory lung unit, the so-called acinus are particularly interesting \cite{Schittny2008,Mund2008,Tsuda2008}.

The acinus is defined as the complex of alveolated airways distal to the terminal bronchiole \cite{Rodriguez1987}, in which the gas-exchange in the lung takes place. 

Our goal is to obtain information about the branching pattern of the airways of the acinus as well as the airflow within it. Tomographic methods, in particular synchrotron radiation based tomographic microscopy can access this kind of information nondestructively and non-invasively. 

In order to visualize and study the tissue septa forming the gas-exchanging alveoli inside the lung, spatial resolution in the order of one micron is required. Full acini---being growing over the postnatal lung development\todo{citation}---are usually larger than the field of view of the microscope, being the latest limited by the chosen optical configuration. Usually, a large field of view resulting in a large sample volume can only be acquired with low magnification and vice-versa. Lab-based micro-computed tomography stations (\micro CT) could potentially be used to study acini, but the resolution of such systems is too low to resolve all alveolar septa. Albeit \micro CT are catching up on this, synchrotron radiation based tomographic microscopy beamlines provide the necessary high resolution combined with unmatched image quality.

Up to now, the price for this was a limited field of view. For instance at the TOMCAT beamline \cite{Stampanoni2007} at the Swiss Light Source, Paul Scherrer Institute, Villigen, Switzerland, the field of view of the 10$\times$ magnification (\SI{0.74}{\micro\meter} voxel size) is limited to 1.52$\times$\SI{1.52}{\milli\meter}, insufficient for imaging whole acini at high resolution.

Increasing the field of view perpendicular to the rotation axis of the sample cannot easily be achieved by placing tomographic datasets next to each other. It is necessary to merge several projection images overlapping the desired field of view prior to tomographic reconstruction. Obviously, according to the sampling theorem increasing the field of view also requires to acquire more projections resulting, finally in an increased acquisition time.

We developed a method to merge several independently acquired sets of projections to increase the field of view of the resulting tomographic dataset. Additionally, through optimization of the number of recorded projections, we established different scanning protocols with an user-defined balance between acquisition time and image quality.

Being the total acquisition time directly linked to the radiation imparted to the sample, it is obvious that such protocols also affect radiation damage and constitute an important optimization tool for radiation sensitive experiments.