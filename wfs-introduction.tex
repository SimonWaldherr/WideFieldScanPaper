%!TEX root = widefieldscan.tex
\svnidlong
{$HeadURL$}
{$LastChangedDate$}
{$LastChangedRevision$}
{$LastChangedBy$}

\ifhtml
\else
\begin{center}
	\fbox{
		\begin{minipage}{.618\columnwidth}
		The section below is versioned at \url{\svnkw{HeadURL}} (last commit @ \svnfileday.\svnfilemonth.\svnfileyear \space \svnfilehour:\svnfileminute, Revision: \svnkw{LastChangedRevision}).
		\end{minipage}
	} 
\end{center}
\fi

\section{Introduction}%

Synchrotron radiation based x-ray tomographic microscopy (SRXTM) is a powerful method for the non-destructive three-dimensional imaging of a broad kind of materials with a resolution on the sub-micrometer scale.

At TOMCAT---the beamline for TOmographic Microscopy and Coherent rAdiology experimenTs~\cite{Stampanoni2007} at the Swiss Light Source (SLS), Paul Scherrer Institute, Villigen, Switzerland---many user groups are presently working in diverse research areas, ranging from biology~\cite{McDonald2009,PerezHuerta2009}, biomedical research~\cite{Schittny2008,Tsuda2008,Heinzer2008} and paleontology~\cite{Gostling2008,Friis2007,Hagadorn2006,Donoghue2006} to material science~\cite{Gallucci2007}, geology~\cite{Carminati2007} to process engineering~\cite{Davenport2007,Vaucher2007}.

\subsection{Background}%
The available field of view of microscopy based imaging methods like synchrotron based tomographic beam lines and lab-based micro-computed tomography (\micro CT) stations is limited by the camera and microscope optics. Commonly a larger field of view resulting in a larger sample volume is traded against a lower magnification. Therefore, at TOMCAT the sample volume is limited to a cylinder with a diameter of \SI{1.46}{\milli\meter} and a height of \SI{1.46}{\milli\meter}, if only one tomogram is taken at a resolution of \SI{1.43}{\micro\meter}. Extending the height of the sample in the direction of the rotational axis of the sample is relatively easy. Multiple tomograms are acquired and subsequently stacked in the direction of the rotational axis. The independently reconstructed three-dimensional stacks of images are just stacked on top of each other in z-direction. 

Samples of a diameter larger than the FOV in the direction perpendicular to the rotational axis require local tomography. During local tomography only the central cylinder where every part of the sample is included on the projectional image of every rotational angle may be reconstructed without artifacts. These artifacts are introduced due to the fact that lateral regions of the sample not included in all projectional images of every rotational angle (i.e.\ partial volume effects). Generally, a local tomography approach is not suitable for studies investigating the development~\cite{Schittny2008,Mund2008} and structure~\cite{Tsuda2008} of lungs using SRXTM, which require high resolution datasets providing a large lateral field of view.

\subsection{Motivation}%
We are studying the structural alterations of the lung parenchyma thoughout  development using SRXTM. In particular we are interested in the development of the acini over the course of the postnatal lung development in mammals. An acinus represents the functional lung unit containing groups of alveoli, where the gas-exchange in the lung takes place.

Until now, the investigation of the three dimensional structure of an acinus was either limited by the resolution of the imaging method (in the case of \micro CT) or the sample volume (in the case of SRXTM). To overcome these limitation, we developed a synchrotron radiation x-ray tomographic microscopy method which combines several tomographic scans into one large three dimensional dataset increasing the scanned volume up to 25 times.

\subsection{Enhancing the Field of View}%
\label{subsec:enhancing the field of view}%
An increase of the field of view parallel to the rotation axis of the sample can be achieved through the stacking of multiple scans on top of each other. For this protocol, the scan time is linearly increasing with the number of scans required to cover the size of the sample. Due to the restrictions implied by the sampling theorem this approach can not directly be used to increase the field of view horizontal to the rotation axis. The sampling theorem states, that we need to acquire an amount of projections $P=D\frac{\pi}{2}$ for a detector width $D$.

As a consequence, when enlarging the field of view perpendicular to the rotation axis it is necessary to record more projections at the lateral parts of the sample compared to the central parts of the sample to equally fulfill the sampling theorem. This leads to an increase in acquisition and post-processing time as compared to a standard scan. We defined and validated multiple scanning protocols to decrease the acquisition time and perfomed three-dimensional reconstructions of datasets with increased field of view of up to \SI{4}{\milli\meter} at a resolution of \SI{1.43}{\micro\meter\per pixel}.