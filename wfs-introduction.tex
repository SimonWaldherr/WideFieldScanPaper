%!TEX root = widefieldscan.tex
\svnidlong
{$HeadURL$}
{$LastChangedDate$}
{$LastChangedRevision$}
{$LastChangedBy$}
%
%\ifhtml
%\else
%\begin{center}
%	\fbox{
%		\begin{minipage}{.618\columnwidth}
%		The section below is versioned at \url{\svnkw{HeadURL}} (last commit @ \svnfileday.\svnfilemonth.\svnfileyear \space \svnfilehour:\svnfileminute, Revision: \svnkw{LastChangedRevision}).
%		\end{minipage}
%	} 
%\end{center}
%\fi
%
\section{Introduction}

%The available field of view of microscopy based imaging methods like synchrotron based tomographic beam lines and lab-based \micro CT stations is limited by the camera and optics. Commonly, a larger field of view resulting in a large sample volume is traded against a lower magnification. As a consequence the investigation of the three dimensional structure of the terminal airway ends in the mammalian lung was up to now either limited by the resolution of the imaging method (in the case of micro-computed tomography (\micro CT)) or the sample volume (in the case of synchrotron radiation based x-ray tomographic microscopy (SRXTM)). We present a method to increase the lateral field of view of SRXTM, a powerful method for the non-destructive three-dimensional imaging of a broad kind of materials at sub-micrometer resolution. 

Using SRXTM, it is possible to study the structural alterations of the lung parenchyma throughout development. We are in particular interested in the development of the acini over the course of the postnatal lung development in mammals. An acinus represents the functional respiratory lung unit containing groups of alveoli, where the gas-exchange in the lung takes place. It is defined as the complex of alveolated airways distal to the terminal bronchiole~\cite{Rodriguez1987}.

At TOMCAT---the beamline for TOmographic Microscopy and Coherent rAdiology experimenTs~\cite{Stampanoni2007} at the Swiss Light Source (SLS), Paul Scherrer Institute, Villigen, Switzerland---the sample volume is limited to a cylinder with a diameter of \SI{1.52}{\milli\meter} and a height of \SI{1.52}{\milli\meter} when using a lens with 10$\times$ magnification and acquiring a tomogram at a voxel size of \SI{1.48}{\micro\meter}. Samples of a diameter larger than the field of view in the direction perpendicular to the rotational axis require local tomography of a region of interest. Generally, a local tomography approach is not suitable for studies investigating the development~\cite{Schittny2008,Mund2008} and structure~\cite{Tsuda2008} of lungs using SRXTM, which require high resolution datasets and a large lateral field of view.

To overcome these limitations, we developed a synchrotron radiation x-ray tomographic microscopy method which combines several tomographic scans into one large three dimensional dataset in order to increase the lateral field of view.

An increase of the field of view parallel to the rotation axis of the sample can be achieved through the stacking of multiple tomographic scans on top of each other. With this scanning mode, the total time needed to acquire a tomogram is linearly increasing with the number of scans required. This scanning mode is also limited to long and thin samples or regions of interest.

Due to the restrictions implied by the sampling theorem this approach can not directly be used to increase the field of view horizontal to the rotation axis. The sampling theorem states, that we need to acquire an amount of projections $P=D\frac{\pi}{2}$ for a detector width $D$~\cite[page 186]{Kak2002}. As a consequence, when enlarging the field of view perpendicular to the rotation axis it is necessary to record more projections at the lateral parts of the sample compared to the central parts of the sample to equally fulfill the sampling theorem. This leads to an increase in acquisition and post-processing time as compared to a standard scan.

The aim of this article is to define and validate multiple scanning protocols to decrease the acquisition time and performed high-quality three-dimensional reconstructions of datasets with increased lateral field of view of up to \SI{7}{\milli\meter} at a voxel size of \SI{1.48}{\micro\meter}. An increase of the lateral field of view permits an unrestricted high-resolution three dimensional view inside the terminal airways only limited by the sample size.