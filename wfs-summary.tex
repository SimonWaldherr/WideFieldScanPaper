% !TEX root = widefieldscan.tex
\svnidlong
{$HeadURL$}
{$LastChangedDate$}
{$LastChangedRevision$}
{$LastChangedBy$}
%
\section{Summary}\label{summary}

A method to increase the lateral field of view of tomographic imaging has been established, which enables the high-resolution tomographic imaging of large samples that are wider than the field of view of the optical setup in multiple semi-automatically combined steps. Tomographic datasets of entire rat lung acini have been acquired with an enhanced field of view using WF-SRXTM.

Different optimized scanning protocols for covering a large field of view have been validated and are now provided for the end-users of the TOMCAT beamline. End-user now have the possibility to choose suitable scanning protocols depending on a balance between acquisition time and expected reconstruction quality. Depending on this balance, we manage to reduce the image acquisition time by \SI{84}{\percent} of the comparable gold standard scan, while keeping the quality of the reconstructed tomographic dataset on a level still permitting automated segmentation of the lung structure and surrounding airspace, as shown in section~\ref{subsec:comparison}. The reduction in acquisition time obviously reduces the time during which the sample is irradiated by the synchrotron radiation and thus reduces the radiation dose inflicted on the sample.