%\documentclass{article}
%\usepackage{tikz}
%
%\begin{document}

\newcommand{\nvar}[2]{%
    \newlength{#1}
    \setlength{#1}{#2}
}

\nvar{\e}{0.79cm}

\begin{figure}[tb]
	\centering
	\begin{tikzpicture}
		%\draw[step=.25,color=gray] (0,0) grid (10,5);
		%\draw(0,0) grid (10,5);
		\draw                (0,0)  -- (4\e,0)  -- (4\e,4\e)    -- (0,4\e)    -- cycle;
		\draw[color=green]   (3\e,0)  -- (7\e,0)  -- (7\e,4\e)  -- (3\e,4\e)  -- cycle;
		\draw[color=blue]    (6\e,0)  -- (10\e,0) -- (10\e,4\e) -- (6\e,4\e)  -- cycle;
		\draw[color=magenta] (9\e,0)  -- (13\e,0) -- (13\e,4\e) -- (9\e,4\e)  -- cycle;
		\draw[color=cyan]    (12\e,0) -- (16\e,0) -- (16\e,4\e) -- (12\e,4\e) -- cycle;
		%\draw (0,0) ellipse (2cm and 1cm) {asdf}
		\draw (2\e,2\e)  node {Scan1};
		\draw (5\e,2\e)  node {Scan2};
		\draw (8\e,2\e)  node {Scan3};
		\draw (11\e,2\e)  node {Scan4};
		\draw (14\e,2\e) node {Scan5};
	\end{tikzpicture}
	\caption{This should ultimatively explain the overlapping scans}
	\label{fig:overlapping scans}
\end{figure}

%\end{document}