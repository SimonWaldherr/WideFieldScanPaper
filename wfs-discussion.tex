%!TEX root = widefieldscan.tex
\svnidlong
{$HeadURL$}
{$LastChangedDate$}
{$LastChangedRevision$}
{$LastChangedBy$}

\ifhtml
\else
\begin{center}
	\fbox{
		\begin{minipage}{.618\columnwidth}
		The section below is versioned at \url{\svnkw{HeadURL}} (last commit @ \svnfileday.\svnfilemonth.\svnfileyear \space \svnfilehour:\svnfileminute, Revision: \svnkw{LastChangedRevision}).
		\end{minipage}
	} 
\end{center}
\fi

\section{Discussion}
\label{sec:Discussion}
We present a method to increase the field of view of parallel beam tomographic imaging methods\todo{discuss what happens for cone-beam} and would like to call this method wide field synchrotron radiation based x-ray tomographic microscopy (WF-SRXTM). We selectively defined different scanning protocols for the optimization of the total imaging time towards the expected imaging quality. This enables the very fast acquisition of lower quality tomographic datasets or acquisition of very high quality datasets in a longer time.

The field of view was increased three-fold by merging projections from three subscans and reconstructing the merged projections using the standard workflow present at the beamline. As a consequence of increasing the field of view, an increased amount of projections have to be aquired to satisfy the sampling theorem. This increased amount of projections lengthens the acquisition time. To overcome this limitation, we defined multiple scanning protocols with a reduced amount of total projections. All these protocols have been evaluated for quality of the resulting reconstructions compared to a gold standard. We have shown that the resulting quality can be simulated prior to scanning, thus giving the end-user a possibility to chose a suited scanning protocol, based on needed scanning time and on an estimation of the quality of the resulting reconstructions.

We have shown that the field of view of parallel beam tomographic end-stations can be increased up to five-fold and have three-dimensionall reconstructed multiple tomograms obtaind with WF-SRXTM. The calculation shown for three subscans (one central and two half ring scans) are expandable to 5 subscans, as has been shown at the end of section~\ref{sec:Results}. The protocols are theoretically expandable for more than the shown 5 subscans, albeit the reconstruction of wide field scans with 7 or more subscans implies tremendous requirements on the data processing infrastructure. The datasets shown in figure~\ref{fig:BvsT} are binned scans resulting in datasets of 1024 slices, each with a size of 2792$\times$2792 pixels at a \SI{8}{\bit} depth. This amounts to a total size of the dataset of approximately \SI{7.5}{\giga\byte}. If we assume an un-binned scan with 7 overlapping subscans, the size of the stitched projections will be around 14000$\times$14000 pixels. The full dataset will consist of 2048 stitched slices with that size, with amounts to a total size of approximately \SI{383}{\giga\byte} for the full dataset. All mentioned datasets have been reconstructed at \SI{8}{\bit}, TOMCAT also offers the possibility to record the tomographic datasets with \SI{16}{\bit} depth, so the full dataset of a wide field scan with a seven-fold increase in field of view with increased bit depth would result in one dataset with a size of \SI{0.75}{\tera\byte}.

Nonetheless the scanning of samples with a five-fold increase in field of view or bigger is interesting, since it would enable the end-user to selectively reconstruct regions of interest (ROI) from large samples with ultra-high resolution. Up to now a two-step process was required to scan ROI from samples larger than the field of view at the high resolution. In a first step, a registered overview scan of the sample at lower resolution and thus large field of view was acquired, then a ROI to be scanned with high resolution was defined in this low-resolution tomographic dataset using a custom-made ROI picker software~\cite{Heinzer2008} and a high resolution local tomography scan with low field of view of this selected ROI was performed at a second allotted beam shift at TOMCAT. This cumbersome two-step process now only has to be used in the case where the highest possible resolution and contrast are required and the limitation in field of view is not important.

One disadvantage of WF-SRXTM compared to the ROI picker method is the extremely big datasets which resulting from the reconstruction of the full resolution projections into reconstructions spanning a large field of view. With the ROI picker method only selective regions of interest selected from a low-resolution tomogram are scanned and reconstructed in high resolution. Since we record all data in a one-step process, it would be possible to integrate partial reconstructions of the full size datasets into the data processing pipeline of TOMCAT. After definition of a ROI to be reconstructed out of the high-resolution wide field dataset, partial sinograms and partial reconstructions could be calculated.