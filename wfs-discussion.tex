%!TEX root = widefieldscan.tex
\svnidlong
{$HeadURL$}
{$LastChangedDate$}
{$LastChangedRevision$}
{$LastChangedBy$}

\ifhtml
\else
\begin{center}
	\fbox{
		\begin{minipage}{.618\columnwidth}
		The section below is versioned at \url{\svnkw{HeadURL}} (last commit @ \svnfileday.\svnfilemonth.\svnfileyear \space \svnfilehour:\svnfileminute, Revision: \svnkw{LastChangedRevision}).
		\end{minipage}
	} 
\end{center}
\fi

\section{Discussion}
We have successfully increased the FOV of the tomographic imaging setup at the TOMCAT beamline. We have selectively defined different scanning protocols for the optimization of the total imaging time towards the expected imaging quality.

The whole scanning process is fully automated; after the user has chosen a defined protocol, all subscans are recorded in an unattended way, without user intervention. After the multiple subscans have been recorded, the stitching of the different amount of projections recorded per subscans into big projections covering the desired FOV is performed with minimal user interaction, i.e.\ the user only has to specify the sample name which is common for all three subscans and the desired output name. The stitching is performed in such a way that the existing reconstruction work flow present at the beamline (as specified by~\cite{Hintermueller2009}) is still applicable and valid, since our  proposed method is applicable as an intermediate step. This is thanks to the modular setup of the beamline setup.

\cbstart
The defined protocols are theoretically expandable to more than the showed 5 subscans, albeit the reconstruction of samples recorded with more subscans implies tremendous requirements on the data processing infrastructure. The datasets shown in figures~\ref{fig:BvsT}--\ref{fig:skel} consist of 1024 slices, each with a size of 2792\(\times\)2792 pixels. This amounts to a total size of the dataset of approximately \SI{7.6}{\giga\byte}. If we assume an un-binned scan with 7 overlapping subscans, the size of the stitched projections will be around 14000\(\times\)14000 pixels. The full dataset will consist of 2048 stitched slices with that size, with amounts to a total size of approximately \SI{383}{\giga\byte} for the full dataset. All mentioned datasets have been recorded at \SI{8}{\bit}, TOMCAT also offers the possibility to record the tomographic datasets with \SI{16}{\bit} depth, so the full dataset of a wide field scan with a seven-fold increase in FOV with this increased bit depth would result in one dataset with a size of \SI{0.75}{\tera\byte}.

Nonetheless the scanning of samples with a more than five-fold increase in FOV is interesting, since it would enable the end-user to selectively reconstructs regions of interest (ROI) from the sample with ultra-high resolution. Up to now a two-step process was required to scan ROI from samples larger than the FOV at the desired resolution. In a first step, a registered overview scan of the sample at lower resolution was acquired, then a ROI to be scanned with high resolution was defined in this low-resolution tomographic dataset using a custom-made ROI picker software~\cite{Heinzer2008} and a high resolution local tomography scan of this selected ROI was performed at a second allotted beam shift at TOMCAT. This cumbersome two-step process is now potentially obsolete, since all the high-resolution data of the sample is obtained during one beam shift in multiple partial scans.
\cbend

\subsection{Still to discuss\ldots}
\begin{itemize}
	\item Problems with insufficient sampling for outer subscans?
	\item Stitching takes long time (incorporation into \verb+prj2sin+ and \verb+sin2rec+ @ beamline?)
	\item idea of partial sinograms and thus partial reconstructions? extraction of ROI from RecoManager and then reconstruct only this Region. Partial stitching of sinograms, padding them to the correct size, correct RotCenter for this sinogram $>$ reconstruction of small size
\end{itemize}