% !TEX root = widefieldscan.tex
\svnidlong
{$HeadURL$}
{$LastChangedDate$}
{$LastChangedRevision$}
{$LastChangedBy$}
%
\section{Discussion}\label{sec:Discussion}

We present a method to laterally increase the field of view of tomographic imaging systems operated in parallel beam geometry and would like to call this method wide field synchrotron radiation based x-ray tomographic microscopy (WF-SRXTM). We selectively defined different scanning protocols for the optimization of the total imaging time towards the expected imaging quality. This enables a very fast acquisition of lower quality tomographic datasets, or acquisition of very high quality datasets in a longer time. Even if the reduction in scanning time does introduce artifacts in the three-dimensional reconstruction, as specified in section~\ref{subsec:comparison}, an automated segmentation of the airway segments in the sample is still possible, even for the protocol with greatly reduced scanning time.

Reducing the scanning time by \SI{84}{\percent} compared to an unoptimized gold standard scan, also greatly reduces radiation dose inflicted on the sample. For this publication, the radiation dose inflicted on the sample was of secondary concern, since our sample was embedded in paraffin and was not used for further investigations\todo{Staining experiments which are planned with Akira: do they also show a problem with radiation damage or is the Osmium-staining the one single problem?}. Nonetheless, such a reduction in radiation dose is the first step to in vivo measurements at TOMCAT. Irradiating living organisms for a prolonged time is not feasible nor desired, especially if biological development is studied, which require multiple exposures of the same individual over a prolonged time-frame\todo{Do we need to be more enthusiastic here?}.

The field of view was increased three-fold by merging projections from three subscans and reconstructing these merged projections using the standard workflow at the TOMCAT beamline~\cite{Hintermueller2009}. As a consequence of increasing the field of view, an increased amount of projections had to be acquired to satisfy the sampling theorem. This increased amount of projections lengthened the acquisition time. To overcome this limitation, we defined multiple scanning protocols with a reduced amount of total projections and thus reduced acquisition time. All these protocols were evaluated for quality of the resulting reconstructions compared to a gold standard. We have shown that the resulting quality can be simulated prior to scanning, thus giving the end-user a possibility to chose a suited scanning protocol, based on the demands for scanning time optimization and quality of the resulting tomographic dataset.

For protocols with an equal amount of total projections, but differing numbers of projections for the central and ring scans we observed a difference in reconstruction quality\todo{Can Xris' morphology-data back this 2D-statement up in 3D?}. Protocols C and D as well as protocols M and N are such protocols and are marked in figure~\ref{fig:NormalizedErrorPlot}. These protocols have the same total number of projections, but different numbers of projections for the central and ring scans (see table~\ref{tab:protocols} for details). This makes it necessary to interpolate projections for the central subscans for both protocol C and M, which decreases the quality of the scan. Additionally, for protocols D and N, the central subscan is oversampled, which does not significantly contribute to the quality of the reconstructions, but makes additional scanning projections necessary. 

The quality $E_{i_{norm}}$ of protocols C and D are not significantly different, and the quality of protocols M and N is comparable\todo{Can Xris' morphology-data back this 2D-statement up in 3D?}. As a consequence, it is thus desirable to favor a protocol where the central scan is not oversampled. Even if this introduces additional computing time for interpolating projections prior to reconstruction, these protocols show an increased quality compared to protocols where the central scan is oversampled. Since an oversampling of the central scan does not add to the total reconstruction quality, this explanation seems natural.

We have shown that the field of view of parallel beam tomographic end-stations can be increased up to five-fold and have three-dimensionally reconstructed multiple tomograms obtained with WF-SRXTM. The protocols are theoretically expandable for more than the shown 5 subscans, although the reconstruction of wide field scans with 7 or more subscans places tremendous requirements on the data processing infrastructure. The datasets shown in figure~\ref{fig:BvsT} are binned scans resulting in datasets of 1024 slices, each with a size of 2792$\times$2792 pixels at a \SI{8}{\bit} depth. This amounts to a total size of the dataset of approximately \SI{7.5}{\giga\byte}. If we assume an un-binned scan with 7 overlapping subscans, the size of the stitched projections will be around 14000$\times$14000 pixels. The full dataset will consist of 2048 stitched slices with that size, which amounts to a total size of approximately \SI{383}{\giga\byte} for the full dataset. All datasets referred to were reconstructed at \SI{8}{\bit}, while TOMCAT also offers the possibility to obtain tomographic datasets with \SI{16}{\bit} depth. A wide field scan with a seven-fold increase in field of view with increased bit depth would result in one dataset with a size of \SI{0.75}{\tera\byte}.

Even if the amount of data to handle is humongous, a wide field scan with a five-fold increase in field of view remains interesting, since it would enable the end-user to selectively reconstruct regions of interest from large samples with ultra-high resolution. Up to now, a two-step process was required to scan such regions from samples larger than the field of view with high resolutions. In a first step, a registered overview scan of the sample at lower resolution, and thus large field of view, was acquired. Then a region of interest to be scanned with high resolution was defined and marked in the low-resolution dataset, using a custom-made ROI picker software~\cite{Heinzer2008}, and a high resolution local tomography scan, with small field of view, was performed at the marked region.

One disadvantage of WF-SRXTM compared with the ROI picker method, are the extremely big datasets that result from the reconstruction of the full resolution projections into reconstructions spanning a large field of view. With the ROI picker method, only selective regions of interest, selected from a low-resolution tomogram, are scanned and reconstructed in high resolution, thus reducing the amount of data recorded. Since we record all data in a one-step process, it would be possible to integrate partial reconstructions of the full size datasets into the data processing pipeline of TOMCAT. After definition of a ROI to be reconstructed out of the high-resolution wide field dataset, partial sinograms and partial reconstructions could be calculated, avoiding a two-step process. 