%!TEX root = widefieldscan.tex
\svnidlong
{$HeadURL$}
{$LastChangedDate$}
{$LastChangedRevision$}
{$LastChangedBy$}
\framebox{Author: \svnauthor|Rev: \svnrev|Last change: \svndate}% - URL: \url{\svnkw{HeadURL}}}
\section{Discussion}
\begin{itemize}
	\item Wide Field Scan is doable
	\item for smallish samples (up to 5 subscans) well suited
	\begin{itemize}
		\item if samples are unbinned, even 5 subscans are too big, R243B06a\_merge (with merged .tifs of a size of 9703$\times$2048 pixels) has not been reconstructed due to RecoManager-\verb+gridrec+-size-Problems.
	\end{itemize}
	\item for bigger samples ($>$ 7 subscans) amount of data is too big to handle (\numprint{2048} rec.tifs with potential sizes of \numprint{14000}$\times$\numprint{14000} px) to be able to reconstruct one full sample with full resolution.
	\item thus partial ROI-reconstruction $>$ extraction of ROI from RecoManager and then reconstruct only this Region. Partial stitching of sinograms, padding them to the correct size, correct RotCenter for this sinogram $>$ reconstruction of small size.
	\item Would potentially solve "problem" with ROI-Picker, where one needs two allotted beam-times to achieve this. First, overview scan, then pick ROI, then detailed scan. With this method, full dataset is obtained in one scan, then partial reconstructions can be done.
\end{itemize}