% !TEX root = widefieldscan.tex
\svnidlong
{$HeadURL$}
{$LastChangedDate$}
{$LastChangedRevision$}
{$LastChangedBy$}
%
\section{Discussion}\label{sec:Discussion}

We present a method to laterally increase the field of view of tomographic imaging systems operated in parallel beam geometry and would like to call this method wide field synchrotron radiation based x-ray tomographic microscopy (WF-SRXTM). We defined scanning protocols for the optimization of the total imaging time versus the expected imaging quality, enabling a very fast acquisition of lower quality tomographic datasets, or acquisition of very high quality datasets in a longer time.

Even if the reduction in scanning time does introduce minor artifacts in the three-dimensional reconstruction, as shown in Figure~\ref{fig:BvsT}, an automated segmentation of the relevant features in the sample is still possible, even for protocols with greatly reduced scanning time.

\cbstart
The introduced artifacts in the three-dimensional reconstruction of the lung tissue are of small scale compared to the alveoli, the smallest structures we would like to visualize. On our scale, the structures which are in the range of our resolution are holes visible in the alveolar septa. Those holes can occur through the three-dimensional reconstruction in places where the alveolar septa are too thin and the globally chosen threshold is too high which leads to the introduction of these holes. However, the observed holes are not only artificial, because \citeasnoun{Kohn1893} described  pores---so-called pores of Kohn---between adjacent alveoli, which can also be seen in rat lungs \cite{Vanmeir1991}. The pores of Kohn have an approximate size in the micrometer scale.

Comparing the reconstructions shown in figure~\ref{fig:BvsT}(d)--(f) we observe a change in size of these pores. The pore size is influenced by both introduced artifacts and the algorithmically chosen threshold in these reconstructions.

Biologically interesting phenomena like emphysematic lung diseases introduce much larger defects in the lung tissue, where the size of the acinus is enlarged and the peripheral airways are collapsed \cite{Weibel2009}. Defects like these would still easily be detectable with an undersampled scan since the introduced artefacts are orders of magnitude smaller than the tissue alterations to be detected.

If other samples are to be observed using the proposed WF-SRXTM method, the desired level of image quality has to defined according to the smallest structure present in the sample to define a suitable reduction in scanning time.

Even if the shorter scanning time introduces minor artifacts in the reconstructed images, such a shorter scanning time can be desired. Since shortening the scanning time also reduces the radiation damage, \cbend even though, for this study, this was not an issue. Obviously, a reduction of the imparted dose to the sample is crucial when radiation sensitive sample are investigated. With a suitable protocol the dose can be reduced by \SI{84}{\percent} (Table~\ref{tab:protocols}), which might be a significant step towards tomographic imaging of sensitive samples using ultra high resolution and enhanced field of view. 

The field of view was increased three-fold by merging projections from three partially overlapping scans and reconstructing these resulting projections using the standard workflow at the TOMCAT beamline (Figure~\ref{fig:wide-field-scan-results}).
\cbstart The high precision of the linear motors used to move the sample stage (resolution better than \SI{1}{\micro\meter} in all three space directions, \SI{0.1}{\micro\meter} accuracy perpendicular to the beam direction \cite{Stampanoni2006a}) permitted an highly reproducible positioning of the lung sample for the consecutive scans.

The sample rotation stage of TOMCAT has a run-out error of less than \SI{1}{\micro\meter} at \SI{100}{\milli\meter} from the rotation surface \cite{Stampanoni2006a}. This precise angular positioning made it possible to merge the projections from the consecutive subscans recorded at the same angular step but differing lateral position into one projection spanning the large field of view.\cbend

As a consequence of the sampling theorem, an increased amount of projections had to be acquired for an increase in the field of view, thus increasing the acquisition time. To overcome this limitation, we defined multiple scanning protocols with a reduced amount of total projections and thus reduced acquisition time and delivered dose (Table~\ref{tab:protocols}). All of these protocols were evaluated for the quality of the resulting reconstructions and compared to a gold standard scan. We have shown that the resulting quality can be simulated prior to scanning and thus provide a tool to choose a suited scanning protocol, based on the demands for scanning time optimization and quality of the resulting tomographic dataset (Figure~\ref{fig:NormalizedErrorPlot}). 

Reducing the amount of projections for the central of the three subscans may be performed with a minor loss of fidelity in the resulting reconstructions. Let us compare protocols D/E and H/I. For protocols E and I we acquired half the amount of projections for the central subscan $\textrm{s}_{2}$ as compared to protocols D and H. In both cases we reduce the scanning time by \SI{17}{\percent}, but keep the quality of the scan on a comparable level (D: $\SI{70}{\percent}\pm3.09$ vs. E: $\SI{80}{\percent}\pm3.01$, H: $\SI{60}{\percent}\pm8.08$ vs. I: $\SI{56}{\percent}\pm3.23$).
% B--C = 13110/15732 = 0.8333, 100%--89%
% D--E = 10925/13110 = 0.8333, 85%--87%
% H--I = 8740/10488 = 0.8303; 78%--80%
We show that the interpolation of missing projections does not introduce relevant errors in the resulting tomographic datasets.

For protocols with an equal amount of total projections, but differing amount of projections for the individual subscans (C/D and M/N) we observed minor differences in reconstruction quality. The qualities $E_{i_{norm}}$ of protocols C and D lie within their respective standard deviation ($\SI{74}{\percent}\pm6.81$ vs. $\SI{70}{\percent}\pm3.09$), and the qualities of protocols M and N are comparable ($\SI{52}{\percent}\pm4.71$ vs. $\SI{42}{\percent}\pm4.78$). Both protocols C and M are scanned without oversampling the central subscan, making interpolation necessary, for protocols D and N we simply stitched the projections of the three subscans. Note that for protocol N we do undersample the outer parts of the sample. When deciding between two protocols with the same amount of total projections, it is thus desirable to favor the protocol where the central scan is not oversampled (i.\,e. choosing protocol C instead of D). Even if this introduces additional computing time to interpolate projections prior to reconstruction, these protocols show an increased quality compared to protocols where the central scan is oversampled. Since an oversampling of the central scan does not add much to the total reconstruction quality and the outer parts of the sample contribute more to the total area of the projections, choosing a protocol where the sampling theorem is satisfied better for those parts of the sample is favorable (i.\,e. favouring protocol M to protocol N).

With the defined protocols we open the possibility for the end-user to choose an acquisition mode suited to fulfill the constraints on number of samples to be scanned within the allocated beamtime and desired quality of the reconstructed datasets.

Additionally, two special use-cases for different protocols are worth mentioning. First, if the user needs a very quick overview over samples at high resolution, a time-saving protocol can be used. This is especially the case, if the integrity of the sample can only be judged with a tomographic scan. Based on the quick scan the right samples for high resolution scans may be selected. It has to be mentioned that a quick overview could---in principle---be obtained with a low-resolution scan, which usually automatically accommodates a larger field of view. However, the resolution of such an overview scan is not always sufficient to detect interesting features in the samples which might be damaged.

We have shown that the field of view of parallel beam tomographic end-stations can be increased up to five-fold and have routinely reconstructed multiple tomograms with a three-fold increase in field of view. The shown acquisition protocols are theoretically expandable for more than five subscans, although the reconstruction of wide field scans with seven or more subscans would require an extremely powerful data processing infrastructure. The datasets shown in Figure~\ref{fig:BvsT} are binned scans resulting in datasets of 1024 slices, each with a size of 2792$\times$2792 pixels at \SI{8}{\bit} gray value depth, which adds up to a total size of the dataset of approximately \SI{7.5}{\giga\byte}. If we assume an unbinned scan with seven overlapping subscans, the size of one stitched projection will be approximately 14000$\times$14000 pixels. The full dataset will consist of 2048 such slices, which would add up to a total size for the full dataset of approximately \SI{383}{\giga\byte}.

Even if the amount of data to handle is huge, a wide field scan with a five-fold increase in field of view remains interesting, since it would enable the end-user to selectively reconstruct only regions of interest from large samples with ultra-high resolution. Up to now, a two-step process was required to scan precisely defined regions from samples larger than the field of view. This process involved the use of different magnifications, two separate beamtimes and a precise registration of the samples between those beamtimes.