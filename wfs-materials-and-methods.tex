% !TEX root = widefieldscan.tex
\svnidlong
{$HeadURL$}
{$LastChangedDate$}
{$LastChangedRevision$}
{$LastChangedBy$}
%
\section{Materials and Methods}
\label{sec:materials and methods}
\subsection{Sample Preparation and Image Acquisition}
Rat lung samples, prepared according to
\ifhtml
	\citet{Tschanz2002} and \citet{Luyet2002}
\else
	\citeasnoun{Tschanz2002} and \citeasnoun{Luyet2002}
\fi
were used as test objects. Briefly, lungs of Sprague-Dawley rats were filled with \SI{2.5}{\percent} glutaraldehyde (\cf{CH2(CH2CHO)2}) in \SI{0.03}{\Molar} potassium-phosphate buffer (pH 7.4) by instillation via tracheotomy at a constant pressure of \SI{20}{\centi\meter} water column. In order to prevent a recoiling of the lung, the pressure was maintained during glutaraldehyde-fixation for a minimum of \SI{2}{\hour}. Subsequently, the lungs were dissected free and immersed in toto in the same fixative at a temperature of \SI{4}{\celsius} for at least \SI{24}{\hour}.

The samples were postfixed with \SI{1}{\percent} osmium tetroxide (\cf{OsO4}) and stained with \SI{4}{\percent} uranyl nitrate (\cf{UO2(NO3)2}) to increase the x-ray absorption contrast, dehydrated in a graded series of ethanol and embedded in paraffin using Histoclear (Merck KGaA, Darmstadt, Germany) as an intermedium. The lung samples were mounted onto standard scanning electron microscopy sample holders (PLANO GmbH, Wetzlar, Germany) using paraffin.

The handling of animals before and during the experiments, as well as the experiments themselves, were approved and supervised by the Swiss Agency for the Environment, Forests and Landscape and the Veterinary Service of the Canton of Bern, Switzerland.

\subsection{SRXTM}
The experiments were performed on the TOMCAT beamline~\cite{Stampanoni2006a} at the Swiss Light Source, Paul Scherrer Institut, Villigen, Switzerland. TOMCAT receives photons from a \SI{2.9}{\tesla} super-bending magnet with a critical energy of \SI{11.1}{\kilo\electronvolt} corresponding to a wavelength of \SI{1.22}{\angstrom}. A double crystal multilayer monochromator covers an energy range between 6 and \SI{45}{\kilo\electronvolt}, with a bandwidth range of a few percent.

\subsubsection{Image acquisition and reconstruction}
\label{seq:Image Acquisition}
The samples were scanned at an x-ray energy of \SI{12.6}{\kilo\electronvolt}, which results in high contrast of the stained tissue compared to the paraffin, and thus a high signal to noise ratio. After penetration of the sample, the x-rays were converted into visible light by a YAG:Ce scintillator (\SI{18}{\micro\meter} thickness, Crismatec Saint-Gobain, Nemours, France). The projection images were magnified by diffraction limited microscope optics and digitized by a high-resolution 2048$\times$2048 pixel CCD camera (pco.2000, PCO AG, Kelheim, Germany) with 14 bit dynamic range. The lung samples were imaged using a 10$\times$ magnification, with 2$\times$2 binning and \SI{175}{\milli\second} exposure time resulting in isotropic voxels with a side length of \SI{1.48}{\micro\meter}.

Projection images $I_{Pr}$, essentially single radiographies of the sample, were recorded at several angular positions between \SI{0}{\degree} and \SI{180}{\degree}. The exact number of angular projections depended on the selected scan protocol, as described in section~\ref{subsec:quality-guided-protocols}. Additionally, a set of dark ($I_{D}$) and flat images ($I_{F}$) were recorded for each protocol. The dark images recorded at the start of the scan capture the noise and dark current of the camera. The flat images recorded at the start and at the end of the scan capture the beam profile and were used to baseline correct the raw projections.

A detailed description of the workflow at TOMCAT and the steps necessary to compute a tomographic data set from a set of angular projections are described by 
\ifhtml
	\citet{Hintermueller2009}
\else
	\citeasnoun{Hintermueller2009}
\fi%
.

\subsubsection{Covering a wide lateral field of view}
For parallel beam geometry, tomographic images are obtained at equidistant angles over a sample rotation of \SI{180}{\degree} as shown in figure~\ref{fig:scanning-possibilities}(a). After reconstruction, the width of the image corresponds to the field of view of the camera.

Samples which are twice as large as the field of view can be imaged using scanning protocols based on a \SI{360}{\degree} off center sample rotation as shown in figure~\ref{fig:scanning-possibilities}(b). Images recorded between \SI{180}{\degree} and \SI{360}{\degree} have to be flipped after acquisition, and the projection images obtained at position $I_{Pr_{x}}$ and $I_{Pr_{x+\SI{180}{\degree}}}$ are stitched together to projection images covering two times the field of view of the camera.

\begin{figure}
	\centering
	\caption{Covering the field of view of differently sized samples with one \SI{180}{\degree} scan (a), one \SI{360}{\degree} scan (b) or---in the case of the so called wide field scanning---with multiple subscans (three subscans, c). The filled segments mark the region of the sample that is covered while scanning the respective positions (Position 1: red/checkerboard, Position 2: green, Position 3: blue/striped).}%
	\ifiucr		
		%\documentclass{article}
%\usepackage[demo]{graphicx}
%\usepackage{subfig}
%\usepackage{tikz}
%	\usetikzlibrary{patterns}
%\usepackage{multirow}
%\usepackage{siunitx}
%\begin{document}
%\begin{figure}
%\centering
%%%%%%%%%%%%%%%%%%%%%%%%%%%%%
\begin{tikzpicture}
	%%%%%%%%%%%%%%%%%%%
	%      grid       %
	%%%%%%%%%%%%%%%%%%%
%		\def\stop{10}
%		\draw [color=gray] (0,0) grid (\stop,\stop);
%		\draw [shade] (0,0) circle (0.25) node {origin};
%		\draw [shade] (\stop,\stop) circle (0.25) node {\stop,\stop};
	%%%%%%%%%%%%%%%%%%%
	%       180       %
	%%%%%%%%%%%%%%%%%%%
		\def\startAtY{9}
		\def\BeamLength{4}
		\def\length{2}
		\node [anchor=north] at (0.5,\startAtY) {a)};
		%camera
			\draw [fill=gray] (0,\startAtY) rectangle (1,\startAtY+1);
			\node [anchor=center] at (0.5,\startAtY+1+.25) {camera};
		% beam
			% inner rays
			\foreach \x in {9,9.1,...,10}
				\draw[gray!50,<-] (1,\x) -- (\BeamLength+1,\x);
			% outer rays
			\foreach \x in {0,1}
				\draw[<-] (1,\startAtY+\x) -- (\BeamLength+1,\startAtY+\x);
			% label				
			\node at (\BeamLength+0.5,\startAtY+1+.25) {beam};
		%sample
			\fill [color=gray,nearly transparent] (0.5*\BeamLength+1,\startAtY+0.5) circle (0.5);
			\fill (0.5*\BeamLength+1,\startAtY+0.5) circle (0.025);
			\draw [thick] (0.5*\BeamLength+1,\startAtY+0.5) circle (0.5);
			\draw [thick,->] (0.5*\BeamLength+1,\startAtY+0.25)   arc (-90:90:0.25);
			\node at (0.5*\BeamLength+1,\startAtY+1+0.25) {sample};
	%%%%%%%%%%%%%%%%%%%
	%       360       %
	%%%%%%%%%%%%%%%%%%%
		\def\startAtY{6}
		\node [anchor=north] at (0.5,\startAtY) {b)};
		%camera
			\draw [fill=gray] (0,\startAtY) rectangle (1,\startAtY+1);
%			\node at (0.5,\startAtY-.25) {camera};
		% beam
			% inner rays
			\foreach \x in {6,6.1,...,7}
				\draw[gray!50,<-] (1,\x) -- (\BeamLength+1,\x);
			% outer rays
			\foreach \x in {0,1}
				\draw[<-] (1,\startAtY+\x) -- (\BeamLength+1,\startAtY+\x);
			% label				
%			\node at (\BeamLength+0.5,\startAtY+1.25) {beam};
		%sample
			\fill [color=gray,nearly transparent] (0.5*\BeamLength+1,\startAtY+1) circle (1);
			\fill (0.5*\BeamLength+1,\startAtY+1) circle (0.025);
			\draw [thick] (0.5*\BeamLength+1,\startAtY+1) circle (1);
			\draw [thick,->] (0.5*\BeamLength+1,\startAtY+0.75) arc (-85:265:0.25);
%			\node at (0.5*\BeamLength+1,\startAtY-0.75) {sample};
	%%%%%%%%%%%%%%%%%%%
	% Wide Field Scan %
	%%%%%%%%%%%%%%%%%%%
		\def\startAtY{2}
		\node [anchor=north] at (0.5,\startAtY) {c)};
		%camera
			\draw [fill=gray] (0,\startAtY) rectangle (1,\startAtY+1);
%			\node at (0.5,\startAtY-.25) {camera};
		% beam
			% inner rays
			\foreach \x in {2,2.1,...,3}
				\draw[gray!50,<-] (1,\x) -- (\BeamLength+1,\x);
			% outer rays
			\foreach \x in {0,1}
				\draw[<-] (1,\startAtY+\x) -- (\BeamLength+1,\startAtY+\x);
			% label				
%			\node at (\BeamLength+0.5,\startAtY+1.25) {beam};
		% shaded samples
		\foreach \y/\position/\color in {\startAtY-0.5/1/red,\startAtY+0.5/2/green,\startAtY+1.5/3/blue}
			{
				\fill [color=\color, nearly transparent] (0.5*\BeamLength+1,\y) circle (1.5);
			}
		\pattern [pattern=checkerboard light gray] (0.5*\BeamLength+1,3) arc (90:-90:1.5) -- ++(0,1) arc (-90:90:0.5);
		\fill [color=red, semitransparent]   (0.5*\BeamLength+1,3) arc (90:-90:1.5) -- ++(0,1) arc (-90:90:0.5);
		\pattern [pattern=horizontal lines gray] (0.5*\BeamLength+1,3+2) arc (90:270:1.5) -- ++(0,1) arc (270:90:0.5);
		\fill [color=blue, semitransparent]  (0.5*\BeamLength+1,3+2) arc (90:270:1.5) -- ++(0,1) arc (270:90:0.5);
		\fill [color=green] (0.5*\BeamLength+1,3-0.5) circle (0.5);
%		\pattern [pattern=vertical lines] (0.5*\BeamLength+1,3-0.5) circle (0.5);
%		\fill [color=red, semitransparent]   (0.5*\BeamLength+1,3) arc (90:-90:1.5) -- ++(0,1) arc (-90:90:0.5);
%		\fill [color=blue, semitransparent]  (0.5*\BeamLength+1,3+2) arc (90:270:1.5) -- ++(0,1) arc (270:90:0.5);
%		\fill [color=green, nearly transparent] (0.5*\BeamLength+1,3-0.5) circle (0.5);
		\foreach \y/\position/\color in {\startAtY-0.5/1/red,\startAtY+0.5/2/green,\startAtY+1.5/3/blue}
			{
				\draw[thick] (0.5*\BeamLength+1,\y) circle (1.5) circle (0.5);
				\draw[thick,->] (0.5*\BeamLength+1,\y-0.25) arc (-90:90:0.25);
				\node at (\BeamLength+2+.005,\y-.005) {Position \position};
				\node [color=\color] at (\BeamLength+2,\y) {Position \position};
				\fill (0.5*\BeamLength+1,\y) circle (0.025);
			}
\end{tikzpicture}
%%%%%%%%%%%%%%%%%%%%%%%%%%%%%	
%\caption{Caption}
%\end{figure}
%\end{document}%
	\else
	\fi
	\label{fig:scanning-possibilities}%
\end{figure}

If a tomographic scan covering a size wider than two field of views is desired, three or more \SI{180}{\degree} scans taken at slightly overlapping positions have to be combined, as shown figure~\ref{fig:scanning-possibilities}(c). The projection images of each subscan overlap slightly to allow an optimal stitching of the multiple projections of each rotational angle into one projection. Using the mean squared difference between adjacent subscan images~\cite{Hintermueller2009}, the position of such a cutline can be calculated for the adjacent subscans and is used to merge the single projections into projections covering the full field of view.

\subsection{Generation of multiple scanning protocols}
The discussed scanning protocol for covering a wide field of view with three subscans is based on the assumption that a sufficient resolution and contrast can be achieved in the tomographic dataset, if the sampling theorem is individually fulfilled for each of the subscans. This results in a set of $i$ subscans with $P_{i}$ projection images. A simple example with $P_{1}=4$ and $P_{2}=P_{3}=8$ is shown in figure~\ref{fig:projections}(a).

To be able to reconstruct the sample, the individual sets of projections need to be stitched to projections spanning the desired field of view. Since each subscan has a different number of projections $P_{i}$, a stitching algorithm has to interpolate projections from adjacent projections (represented by the black dashed lines in figure~\ref{fig:projections}(b)). Alternatively, an equal amount of projections can to be acquired for all the subscans, leading to oversampling for the central scanning position and prolonging the acquisition time.

\begin{figure}
	\centering
	\caption{Setup with three \SI{180}{\degree} scans; one central (green) and two lateral (red and blue, respectively). In this drawing, four projections for the central and eight projections for each of the lateral scans have been chosen. The colors of the three positions correspond to the colors shown in figure~\ref{fig:scanning-possibilities}(c). Panel a): scanned projections, panel b): scanned projections and additional interpolated projections (dashed) required to merge all projections.}
	\ifiucr
		%\documentclass{article}
%\usepackage[demo]{graphicx}
%\usepackage{subfig}
%\usepackage{tikz}
%\usepackage{multirow}
%\usepackage{siunitx}
%\begin{document}
%\begin{figure}
%\centering
%%%%%%%%%%%%%%%%%%%%%%%%%%%%%
\def\radius{1}%
\def\gap{0.05}%
\begin{tikzpicture}[scale=.75]%
	\foreach \ang in {0,45,...,359}%
		{%
		\draw [ultra thick] (\ang:0) -- (\ang:\radius);%
		}%
	\foreach \ang in {0,45,...,359}%
		{%
		\draw [very thick, color=yellow, shorten >=0.25pt] (\ang:0) -- (\ang:\radius);%
		}%
	\foreach \ang in {0,22.5,...,179}%
		{%
		\draw [ultra thick] (\ang:\radius+\gap) -- (\ang:3*\radius+\gap);%
		\draw [very thick, color=magenta, shorten >=0.25pt,shorten <=0.25pt] (\ang:\radius+\gap) -- (\ang:3*\radius+\gap);%
		}%
	\foreach \ang in {180,202.5,...,359}%
		{%
		\draw [ultra thick] (\ang:\radius+\gap) -- (\ang:3*\radius+\gap);%		
		\draw [very thick, color=cyan, shorten >=0.25pt,shorten <=0.25pt] (\ang:\radius+\gap) -- (\ang:3*\radius+\gap);%
		}%
	\node [anchor=south west] at (-3.05,-3.05) {(a)};
\end{tikzpicture}
%%%%%%%%%%%%%%%%%%%%%%%%%%%%%	
%\caption{Projection Setup}
%\end{figure}
%\end{document}%
		%\documentclass{article}
%\usepackage[demo]{graphicx}
%\usepackage{subfig}
%\usepackage{tikz}
%\usepackage{multirow}
%\usepackage{siunitx}
%\begin{document}
%\begin{figure}
%\centering
%%%%%%%%%%%%%%%%%%%%%%%%%%%%%
\def\radius{1}
\def\gap{0.05}
\begin{tikzpicture}[ultra thick]
	\foreach \ang in {22.5,67.5,...,359}%
		{%
		\draw [dashed] (\ang:0) -- (\ang:\radius);%
		}%			
	\foreach \ang in {0,45,...,359}%
		{%
		\draw [color=green] (\ang:0) -- (\ang:\radius);%
		}%
	\foreach \ang in {0,22.5,...,179}%
		{%
		\draw [color=red] (\ang:\radius+\gap) -- (\ang:\radius+\radius+\gap);%
		}%
	\foreach \ang in {180,202.5,...,359}%
		{%
		\draw [color=blue] (\ang:\radius+\gap) -- (\ang:\radius+\radius+\gap);%
		}%
\end{tikzpicture}
%%%%%%%%%%%%%%%%%%%%%%%%%%%%%	
%\caption{Projection Setup}
%\end{figure}
%\end{document}

	\else
	\fi
	\label{fig:projections}
\end{figure}

\subsubsection{Reduction of acquisition time and radiation dose}
\label{subsubsec:reduction-of-acquisition-time}
Since the total acquisition time per sample linearly scales with the total amount of recorded projections for all subscans, oversampling is generally avoided, to reduce the total amount of beam time used for one sample. Our goal was to find a good compromise between scanning time and image quality. Acquisition protocols with varying number of projections and thus varying acquisition time were designed in order to reduce the radiation dose on the sample.

To simplify the interpolation and merging of the projections from each subscan, we selected protocols where the fraction of the amount of projections of the inner ($P_{inner}$) to the outer subscans ($P_{outer}$) is always a multiple of two%: $\frac{P_{outer}}{P_{inner}} \bmod 2 = 0$
. In the simple case shown in figure~\ref{fig:projections}, 4 projections for the central and 8 projections for each of the lateral scans would be acquired, thus avoiding oversampling the central part of the sample. To be able to merge the projection images, we need to interpolate 4 projections (dashed) prior to stitching. If the number of projections acquired for the ring scans was not two times the number of projections of the central scan, the intermediate projections could not be interpolated equally from the rest of the projections in a 1:1-way. Instead, they would need to be interpolated in a 1:2 or 1:3-way, which would introduce additional processing overhead.

To further reduce the total scanning time, we developed different scanning protocols related to a gold standard protocol. This gold standard protocol covers the desired field of view while fulfilling the sampling theorem in all parts of the field of view, as shown in figure~\ref{fig:SubScan-Setup}(a). For this example, we are only showing one slice of the reconstructed dataset and can thus assume a detector with a height of 1 pixel. In this case we would like to achieve a field of view of 3072 pixels. The dark gray circle, shown in the aforementioned figure is the field of view that can be fully covered using a detector with a diameter of 3072 pixels. The rest of the reconstructed slice (in light grey) would be either missing or show artifacts, depending on the chosen reconstruction algorithm.

Using a detector with a size of 1024 pixels, this desired field of view could be covered with nine independent scans. Such an approach would require nine independent reconstructions and stitching of those nine tomographic datasets into one dataset covering the full field of view. Stitching 9 smaller reconstructions like this would also introduce artifacts from the reconstruction algorithm (contained in the light gray parts filling the room between the square and the circle) which would lie inside the sample to be imaged.

While the chosen field of view of 3072$\times$3072 pixels can be covered using a detector of the size of 3072 pixels in one scan, we can cover the desired field of view with a much smaller detector, using a scanning protocol with three subscans from which we obtain merged projections. Figure~\ref{fig:SubScan-Setup}(b) shows how the desired field of view of 3072 pixels can be covered with a wide-field scan, composed of one central and two half ring-scans, recorded with a detector with a size of 1024 pixels. A further increase in the field of view can be obtained by simple iteration. Figures~\ref{fig:SubScan-Setup}(c)--(f) show such a setup for a five- or seven-fold increase.

\begin{figure}
	\centering
	\caption{Setup for different field of views. %
		(a) Desired field of view of 3072 pixel diameter. %
		(b) Gold standard scanning protocol for covering the desired field of view of panel (a) with merged projections from one central and two half ring scans ($r_{1}$ and $r_{2}$). %
		(c) Desired field of view of 5120 pixel diameter. %
		(d) Gold standard scanning protocol for covering the desired field of view of panel (c) with merged projections from one central and four half ring scans ($r_{1}$--$r_{4}$). %
		(e) Desired field of view of 7168 pixel diameter. %
		(f) Gold standard scanning protocol for covering the desired field of view of panel (e) with merged projections from one central and six half ring scans ($r_{1}$--$r_{6}$).}%
	\ifiucr		
		%\documentclass{article}
%\usepackage{subfig}
%\usepackage{tikz}
%\begin{document}
%\begin{figure}
%\centering
%%%%%%%%%%%%%%%%%%%%%%%%%%%%% 3 SUBSCANS %%%%%%%%%%%%%%%%%%%%%%%%%%%%%
\def\scale{0.65} % -> (3*.65=1.95)
\def\size{3}
\subfloat[FOV to be covered]{%
	\label{subfig:fovneed3}%
	\begin{tikzpicture}[scale=\scale]
	%	\draw [dashed] (-1,-1) grid (7,7);
		\draw [fill=gray!25] (0,0) rectangle (2*\size,2*\size);
		\fill [semitransparent] (\size,\size) circle (\size);
		\draw (\size,\size) circle (\size);
		\draw [white,ultra thick,<->] (0,\size) -- node [above] {3072 px} (2*\size,\size);
	%	\draw [step=2] (0,0) grid (6,6);
	\end{tikzpicture}%
}\hfill
\subfloat[Gold Standard; covering the FOV with 9 independently reconstructed small scans.]{%
	\label{subfig:goldstandard3}%
	\begin{tikzpicture}[scale=\scale]
	%	\draw [dashed] (-1,-1) grid (7,7);
		\fill [color=gray!25] (0,0) rectangle (2*\size,2*\size);
		\foreach \x in {1,2,3}
			\foreach \y in {1,2,3}
				\draw [fill=gray] (2*\x-1,2*\y-1) circle (1) node {\x,\y};
		\fill [semitransparent] (\size,\size) circle (\size);
		\draw (\size,\size) circle (\size);
		\draw [step=2] (0,0) grid (2*\size,2*\size);
		\draw [white,ultra thick,<->] (\size-1,1.1*\size) -- node [above] {1024 px} (\size+1,1.1*\size);
	\end{tikzpicture}%
}\hfill
\subfloat[Covering the FOV with merged projections from one central and two ring scans.]{%
	\label{subfig:protocol3}%
	\begin{tikzpicture}[scale=\scale]
%		\draw [dashed] (-1,-1) grid (7,7);
		\draw [fill=gray!25] (0,0) rectangle (2*\size,2*\size);
		\fill [semitransparent] (\size,\size) circle (\size);
		\foreach \r in {1,3}
			\draw (\size,\size) circle (\r);
		\draw (0,\size) -- (\size-1,\size);
		\draw (\size+1,\size) -- (2*\size,\size);
		\node at (\size,1) {r1};
		\node at (\size,3) {central};
		\node at (\size,5) {r2};
		\def\angle{155}
		\draw [white,ultra thick,<->] (\size,\size) +(\angle:1) -- node [sloped,midway,above] {1024 px} +(\angle:3); 
	\end{tikzpicture}
}
%%%%%%%%%%%%%%%%%%%%%%%%%%%%% 3 SUBSCANS %%%%%%%%%%%%%%%%%%%%%%%%%%%%%  
%%%%%%%%%%%%%%%%%%%%%%%%%%%%% 5 SUBSCANS %%%%%%%%%%%%%%%%%%%%%%%%%%%%%
\def\scale{0.39} % 1.95/5
\def\size{5}
\subfloat[FOV to be covered]{%
	\label{subfig:fovneed5}%
	\begin{tikzpicture}[scale=\scale]
	%	\draw [dashed] (-1,-1) grid (7,7);
		\draw [fill=gray!25] (0,0) rectangle (2*\size,2*\size);
		\fill [semitransparent] (\size,\size) circle (\size);
		\draw (\size,\size) circle (\size);
		\draw [white,ultra thick,<->] (0,\size) -- node [above] {5120 px} (2*\size,\size);
	%	\draw [step=2] (0,0) grid (6,6);
	\end{tikzpicture}%
}\hfill
\subfloat[Gold Standard; covering the FOV with 25 independently reconstructed small scans.]{%
	\label{subfig:goldstandard5}%
	\begin{tikzpicture}[scale=\scale]
	%	\draw [dashed] (-1,-1) grid (7,7);
		\fill [color=gray!25] (0,0) rectangle (2*\size,2*\size);
		\foreach \x in {1,2,3,4,5}
			\foreach \y in {1,2,3,4,5}
				\draw [fill=gray] (2*\x-1,2*\y-1) circle (1) node {\x,\y};
		\fill [semitransparent] (\size,\size) circle (\size);
		\draw (\size,\size) circle (\size);
		\draw [step=2] (0,0) grid (2*\size,2*\size);
		\draw [white,ultra thick,<->] (\size-1,1.1*\size) -- node [above] {1024 px} (\size+1,1.1*\size);
	\end{tikzpicture}%
}\hfill
\subfloat[Covering the FOV with merged projections from one central and four ring scans.]{%
	\label{subfig:protocol5}%
	\begin{tikzpicture}[scale=\scale]
%		\draw [dashed] (-1,-1) grid (7,7);
		\draw [fill=gray!25] (0,0) rectangle (2*\size,2*\size);
		\fill [semitransparent] (\size,\size) circle (\size);
		\foreach \r in {1,3,5}
			\draw (\size,\size) circle (\r);
		\draw (0,\size) -- (\size-1,\size);
		\draw (\size+1,\size) -- (2*\size,\size);
		\node at (\size,1) {r3};
		\node at (\size,3) {r1};
		\node at (\size,5) {central};
		\node at (\size,7) {r2};
		\node at (\size,9) {r4};
		\def\angle{155}
		\draw [white,ultra thick,<->] (\size,\size) +(\angle:1) -- node [sloped,midway,above] {1024 px} +(\angle:3); 
	\end{tikzpicture}
}
%%%%%%%%%%%%%%%%%%%%%%%%%%%%% 5 SUBSCANS %%%%%%%%%%%%%%%%%%%%%%%%%%%%%
%%%%%%%%%%%%%%%%%%%%%%%%%%%%% 7 SUBSCANS %%%%%%%%%%%%%%%%%%%%%%%%%%%%%
\def\scale{0.27857142857142857142857142857143} % 1.95/7
\def\size{7}
\subfloat[FOV to be covered]{%
	\label{subfig:fovneed7}%
	\begin{tikzpicture}[scale=\scale]
	%	\draw [dashed] (-1,-1) grid (7,7);
		\draw [fill=gray!25] (0,0) rectangle (2*\size,2*\size);
		\fill [semitransparent] (\size,\size) circle (\size);
		\draw (\size,\size) circle (\size);
		\draw [white,ultra thick,<->] (0,\size) -- node [above] {7168 px} (2*\size,\size);
	%	\draw [step=2] (0,0) grid (6,6);
	\end{tikzpicture}%
}\hfill
\subfloat[Gold Standard; covering the FOV with 49 independently reconstructed small scans.]{%
	\label{subfig:goldstandard7}%
	\begin{tikzpicture}[scale=\scale]
	%	\draw [dashed] (-1,-1) grid (7,7);
		\fill [color=gray!25] (0,0) rectangle (2*\size,2*\size);
		\foreach \x in {1,2,3,4,5,6,7}
			\foreach \y in {1,2,3,4,5,6,7}
				\draw [fill=gray] (2*\x-1,2*\y-1) circle (1) node {\x,\y};
		\fill [semitransparent] (\size,\size) circle (\size);
		\draw (\size,\size) circle (\size);
		\draw [step=2] (0,0) grid (2*\size,2*\size);
		\draw [white,ultra thick,<->] (\size-1,1.1*\size) -- node [above] {1024 px} (\size+1,1.1*\size);
	\end{tikzpicture}%
}\hfill
\subfloat[Covering the FOV with merged projections from one central and six ring scans.]{%
	\label{subfig:protocol7}%
	\begin{tikzpicture}[scale=\scale]
%		\draw [dashed] (-1,-1) grid (7,7);
		\draw [fill=gray!25] (0,0) rectangle (2*\size,2*\size);
		\fill [semitransparent] (\size,\size) circle (\size);
		\foreach \r in {1,3,5,7}
			\draw (\size,\size) circle (\r);
		\draw (0,\size) -- (\size-1,\size);
		\draw (\size+1,\size) -- (2*\size,\size);
		\node at (\size,1) {r5};
		\node at (\size,3) {r3};
		\node at (\size,5) {r1};
		\node at (\size,7) {central};
		\node at (\size,9) {r2};
		\node at (\size,11) {r4};
		\node at (\size,13) {r6};
		\def\angle{155}
		\draw [white,ultra thick,<->] (\size,\size) +(\angle:1) -- node [sloped,midway,above] {1024 px} +(\angle:3); 
	\end{tikzpicture}
}
%%%%%%%%%%%%%%%%%%%%%%%%%%%%% 7 SUBSCANS %%%%%%%%%%%%%%%%%%%%%%%%%%%%%
%%%%%%%%%%%%%%%%%%%%%%%%%%%%%	
%\caption{Projection Setup}
%\end{figure}
%\end{document}
	\else
	\fi
	\label{fig:SubScan-Setup}
\end{figure}

\subsection{Wide Field Scanning Pipeline}
\label{subsec:wfs-setup}
A MATLAB-script (MATLAB\textsuperscript{\textregistered} 7.6.0.324 (R2008a), The MathWorks, Inc.) reads several parameters like desired field of view, detector width, magnification and binning and calculates the number of projections needed to satisfy the sampling theorem for the chosen field of view. This gold-standard protocol is used to compute a set of acquisition protocols with reduced amount of projections.

Using a Shepp-Logan phantom~\cite{Shepp1974} as base image, a simulated scan and subsequent reconstruction is calculated for each protocol of this set. The expected reconstruction quality is calculated, based on the difference between the simulated reconstructions and the gold standard, which is equal to the original phantom. This calculated reconstruction quality is then plotted against the acquisition time, which enables the user to chose a suitable protocol balancing expected image quality and acquisition time.

After the user has chosen a suitable protocol out of the presented set, a file containing the details of the scan is written to disk. This file contains the parameters needed to scan this protocol: the number of required subscans, the positions of the sample, the number of projections and the start and stop angles of the rotation for each subscan. A custom Python-script (\url{http://python.org/}) parses the file containing the details and interacts with the EPICS-System (Experimental Physics and Industrial Control System, Argonne National Laboratory, Argonne, USA, \url{http://www.aps.anl.gov/epics/}), which itself is used to control the hardware of the TOMCAT beamline, enabling an automated, unattended batch scan of all the desired subscans.

After the multiple subscans have been recorded, the stitching of the projections recorded per subscans into projections covering the desired field of view is performed with minimal user interaction using a second MATLAB-script. Further processing of the merged projections such as sinogram generation and reconstruction into the tomographic dataset are integrated into the TOMCAT data-processing pipeline, as specified by
\ifhtml
	\citet{Hintermueller2009}%
\else
	\citeasnoun{Hintermueller2009}%
\fi%
.

\subsection{Batch Acquisition of the Protocols}
Multiple scanning protocols were defined in a way that the total amount of recorded projections for the different protocols were linearly scaled down compared to a gold standard scan (for details, see section~\ref{subsec:quality-guided-protocols}). All parameters of each protocol, such as sample-position in relation to the beam, rotation angles and amount of projections to obtain for each of the individual subscans were set in a preference-file, generated using the aforementioned MATLAB-script. All 19 protocols (B--T) were scanned as a batch scan without manual intervention, permitting a direct comparison of the reconstructed datasets only limited by the repositioning inaccuracies of the TOMCAT beamline.

After acquisition of the three subscans per protocol, custom MATLAB functions read the parameters of the single subscans (e.g.\ sample name, amount of subscans, amount of dark and flat images) as well as the desired output-name and -suffix, and performed all necessary calculations, including: loading of the correct projections from each subscan; normalizing; interpolation; cutline detection; correct stitching of the images into wide field projections, and writing these merged projections to disk.

The merged projections were rearranged into sinograms, where the $n$\textsuperscript{th} sinogram is composed of the $n$\textsuperscript{th} line of every corrected projection. The $n$\textsuperscript{th} slice of the tomographic scan was reconstructed from the $n$\textsuperscript{th} sinogram using an FFT-based regridding algorithm~\cite{Dowd1999}. The 19 tomographic datasets were reconstructed on a 20-node server farm composed of five \SI{64}{\bit} Opteron machines with 4 cores and \SI{8}{\giga\byte} RAM each. The reconstructions resulted in an image stack covering a large sample volume of 2793$\times$2793$\times$1024 pixels, a nine-fold increase from the standard volume of 1024$\times$1024$\times$1024 pixels for one conventional scan.